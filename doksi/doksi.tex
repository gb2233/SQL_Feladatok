% Options for packages loaded elsewhere
\PassOptionsToPackage{unicode}{hyperref}
\PassOptionsToPackage{hyphens}{url}
\PassOptionsToPackage{dvipsnames,svgnames*,x11names*}{xcolor}
%
\documentclass[
]{article}
\usepackage{lmodern}
\usepackage{amssymb,amsmath}
\usepackage{ifxetex,ifluatex}
\ifnum 0\ifxetex 1\fi\ifluatex 1\fi=0 % if pdftex
  \usepackage[T1]{fontenc}
  \usepackage[utf8]{inputenc}
  \usepackage{textcomp} % provide euro and other symbols
\else % if luatex or xetex
  \usepackage{unicode-math}
  \defaultfontfeatures{Scale=MatchLowercase}
  \defaultfontfeatures[\rmfamily]{Ligatures=TeX,Scale=1}
  \setmainfont[]{DejaVu Sans}
  \setmonofont[]{DejaVu Sans Mono}
\fi
% Use upquote if available, for straight quotes in verbatim environments
\IfFileExists{upquote.sty}{\usepackage{upquote}}{}
\IfFileExists{microtype.sty}{% use microtype if available
  \usepackage[]{microtype}
  \UseMicrotypeSet[protrusion]{basicmath} % disable protrusion for tt fonts
}{}
\makeatletter
\@ifundefined{KOMAClassName}{% if non-KOMA class
  \IfFileExists{parskip.sty}{%
    \usepackage{parskip}
  }{% else
    \setlength{\parindent}{0pt}
    \setlength{\parskip}{6pt plus 2pt minus 1pt}}
}{% if KOMA class
  \KOMAoptions{parskip=half}}
\makeatother
\usepackage{xcolor}
\IfFileExists{xurl.sty}{\usepackage{xurl}}{} % add URL line breaks if available
\IfFileExists{bookmark.sty}{\usepackage{bookmark}}{\usepackage{hyperref}}
\hypersetup{
  pdftitle={SQL projektfeladat dokumentáció},
  colorlinks=true,
  linkcolor=blue,
  filecolor=Maroon,
  citecolor=Blue,
  urlcolor=Blue,
  pdfcreator={LaTeX via pandoc}}
\urlstyle{same} % disable monospaced font for URLs
\usepackage[margin=2cm,a4paper]{geometry}
\usepackage{color}
\usepackage{fancyvrb}
\newcommand{\VerbBar}{|}
\newcommand{\VERB}{\Verb[commandchars=\\\{\}]}
\DefineVerbatimEnvironment{Highlighting}{Verbatim}{commandchars=\\\{\}}
% Add ',fontsize=\small' for more characters per line
\usepackage{framed}
\definecolor{shadecolor}{RGB}{248,248,248}
\newenvironment{Shaded}{\begin{snugshade}}{\end{snugshade}}
\newcommand{\AlertTok}[1]{\textcolor[rgb]{1.00,0.00,0.00}{\textbf{#1}}}
\newcommand{\AnnotationTok}[1]{\textcolor[rgb]{0.38,0.63,0.69}{\textbf{\textit{#1}}}}
\newcommand{\AttributeTok}[1]{\textcolor[rgb]{0.49,0.56,0.16}{#1}}
\newcommand{\BaseNTok}[1]{\textcolor[rgb]{0.25,0.63,0.44}{#1}}
\newcommand{\BuiltInTok}[1]{#1}
\newcommand{\CharTok}[1]{\textcolor[rgb]{0.25,0.44,0.63}{#1}}
\newcommand{\CommentTok}[1]{\textcolor[rgb]{0.61,0.61,0.61}{#1}}
\newcommand{\CommentVarTok}[1]{\textcolor[rgb]{0.38,0.63,0.69}{\textbf{\textit{#1}}}}
\newcommand{\ConstantTok}[1]{\textcolor[rgb]{0.53,0.00,0.00}{#1}}
\newcommand{\ControlFlowTok}[1]{\textcolor[rgb]{0.00,0.44,0.13}{\textbf{#1}}}
\newcommand{\DataTypeTok}[1]{\textcolor[rgb]{0.56,0.13,0.00}{#1}}
\newcommand{\DecValTok}[1]{\textcolor[rgb]{0.25,0.63,0.44}{#1}}
\newcommand{\DocumentationTok}[1]{\textcolor[rgb]{0.73,0.13,0.13}{\textit{#1}}}
\newcommand{\ErrorTok}[1]{\textcolor[rgb]{1.00,0.00,0.00}{\textbf{#1}}}
\newcommand{\ExtensionTok}[1]{#1}
\newcommand{\FloatTok}[1]{\textcolor[rgb]{0.25,0.63,0.44}{#1}}
\newcommand{\FunctionTok}[1]{\textcolor[rgb]{0.02,0.16,0.49}{#1}}
\newcommand{\ImportTok}[1]{#1}
\newcommand{\InformationTok}[1]{\textcolor[rgb]{0.38,0.63,0.69}{\textbf{\textit{#1}}}}
\newcommand{\KeywordTok}[1]{\textcolor[rgb]{0.00,0.44,0.13}{\textbf{#1}}}
\newcommand{\NormalTok}[1]{#1}
\newcommand{\OperatorTok}[1]{\textcolor[rgb]{0.40,0.40,0.40}{#1}}
\newcommand{\OtherTok}[1]{\textcolor[rgb]{0.00,0.44,0.13}{#1}}
\newcommand{\PreprocessorTok}[1]{\textcolor[rgb]{0.74,0.48,0.00}{#1}}
\newcommand{\RegionMarkerTok}[1]{#1}
\newcommand{\SpecialCharTok}[1]{\textcolor[rgb]{0.25,0.44,0.63}{#1}}
\newcommand{\SpecialStringTok}[1]{\textcolor[rgb]{0.73,0.40,0.53}{#1}}
\newcommand{\StringTok}[1]{\textcolor[rgb]{0.25,0.44,0.63}{#1}}
\newcommand{\VariableTok}[1]{\textcolor[rgb]{0.10,0.09,0.49}{#1}}
\newcommand{\VerbatimStringTok}[1]{\textcolor[rgb]{0.25,0.44,0.63}{#1}}
\newcommand{\WarningTok}[1]{\textcolor[rgb]{0.38,0.63,0.69}{\textbf{\textit{#1}}}}
\setlength{\emergencystretch}{3em} % prevent overfull lines
\providecommand{\tightlist}{%
  \setlength{\itemsep}{0pt}\setlength{\parskip}{0pt}}
\setcounter{secnumdepth}{-\maxdimen} % remove section numbering
\usepackage{sectsty}
\sectionfont{\clearpage}
\usepackage{fancyvrb,newverbs,xcolor}

\definecolor{Light}{gray}{.90}

\let\oldtexttt\texttt
\renewcommand{\texttt}[1]{
  \colorbox{Light}{\oldtexttt{#1}}
}
\usepackage{enumitem}
\usepackage{amsfonts}

% level one
\setlist[itemize,1]{label=$\bullet$}
% level two
\setlist[itemize,2]{label=$\circ$}
% level three
\setlist[itemize,3]{label=$\star$}
\usepackage{hyperref}

\hypersetup{
  pdftitle={SQL Feladat Dokumentáció},
  pdfauthor={Gulyás Bálint},
  pdfsubject={Dokumentáció}
}

\title{SQL projektfeladat dokumentáció}
\date{}

\begin{document}
\maketitle

\hypertarget{bookings-tuxe1bla-elkuxe9szuxedtuxe9se1}{%
\section[Bookings tábla elkészítése]{\texorpdfstring{Bookings tábla
elkészítése\footnote{A teljes scriptek a mellékelt SSMS solution-ben
  találhatóak}}{Bookings tábla elkészítése}}\label{bookings-tuxe1bla-elkuxe9szuxedtuxe9se1}}

\hypertarget{kezdeti-adatok-importuxe1luxe1sa}{%
\subsection{Kezdeti adatok
importálása}\label{kezdeti-adatok-importuxe1luxe1sa}}

Az európai városoknevek és az országok ISO kód listái külső
forrásból\footnote{\href{https://datahub.io/core/country-list?fbclid=IwAR1Kapllmzc9sthOPNstkF23BomEfkQeLyivSLC2joxgqqdsoksLm9FP3qw}{ISO
  Kódok},
  \href{http://worldpopulationreview.com/continents/cities-in-europe/?fbclid=IwAR2zDepceQtlVAJOgoXHoPPrG6RKVhriUGgWYut_feKryAoLXVCs36y-Ip0}{Városok}}
lettek importálva két segédtáblába.

Countries tábla\footnote{\emph{} helyettesítendő a kívánt fájl helyével}
:

\begin{Shaded}
\begin{Highlighting}[]
\KeywordTok{CREATE} \KeywordTok{TABLE}\NormalTok{ Countries(}
\NormalTok{    [asciiname] }\DataTypeTok{VARCHAR}\NormalTok{(}\DecValTok{255}\NormalTok{) }\KeywordTok{NOT} \KeywordTok{NULL} \KeywordTok{PRIMARY} \KeywordTok{KEY}\NormalTok{,}
\NormalTok{    [country] }\DataTypeTok{VARCHAR}\NormalTok{(}\DecValTok{255}\NormalTok{) }\KeywordTok{NOT} \KeywordTok{NULL}\NormalTok{,}
\NormalTok{    [population] }\DataTypeTok{int} \KeywordTok{NOT} \KeywordTok{NULL}
\NormalTok{)}
\KeywordTok{BULK} \KeywordTok{INSERT}\NormalTok{ Countries}
\KeywordTok{FROM} \StringTok{'<Elérési Útvonal>\textbackslash{}countries.csv'}
\KeywordTok{WITH}\NormalTok{ (FORMAT}\OperatorTok{=}\StringTok{'CSV'}\NormalTok{,}
\NormalTok{    FIRSTROW}\OperatorTok{=}\DecValTok{2}\NormalTok{,}
\NormalTok{    FIELDTERMINATOR }\OperatorTok{=} \StringTok{'|'}\NormalTok{,}
\NormalTok{    ROWTERMINATOR }\OperatorTok{=} \StringTok{'0x0a'}\NormalTok{)}
\end{Highlighting}
\end{Shaded}

Iso2Codes tábla\footnote{\emph{} helyettesítendő a kívánt fájl helyével}
:

\begin{Shaded}
\begin{Highlighting}[]
\KeywordTok{CREATE} \KeywordTok{TABLE}\NormalTok{ Iso2Codes(}
\NormalTok{    [Code] }\DataTypeTok{VARCHAR}\NormalTok{(}\DecValTok{2}\NormalTok{) }\KeywordTok{NOT} \KeywordTok{NULL}\NormalTok{,}
\NormalTok{    [Name] }\DataTypeTok{VARCHAR}\NormalTok{(}\DecValTok{255}\NormalTok{) }\KeywordTok{NOT} \KeywordTok{NULL} \KeywordTok{PRIMARY} \KeywordTok{KEY}
\NormalTok{)}
\KeywordTok{BULK} \KeywordTok{INSERT}\NormalTok{ Iso2Codes}
\KeywordTok{FROM} \StringTok{'<Elérési Útvonal>\textbackslash{}iso2.csv'}
\KeywordTok{WITH}\NormalTok{ (FORMAT}\OperatorTok{=}\StringTok{'CSV'}\NormalTok{,}
\NormalTok{    FIRSTROW}\OperatorTok{=}\DecValTok{2}\NormalTok{,}
\NormalTok{    FIELDTERMINATOR }\OperatorTok{=} \StringTok{'|'}\NormalTok{,}
\NormalTok{    ROWTERMINATOR }\OperatorTok{=} \StringTok{'0x0a'}\NormalTok{)}
\end{Highlighting}
\end{Shaded}

\hypertarget{kieguxe9szuxedtux151-adathalmazok-luxe9trehozuxe1sa}{%
\subsection{Kiegészítő adathalmazok
létrehozása}\label{kieguxe9szuxedtux151-adathalmazok-luxe9trehozuxe1sa}}

A kiegészítő adatok ideiglenes táblában vannak tárolva a
felhasználásukig, az átláthatóság kedvéért.

Az ISO kódok hozzárendelése csak az európai országokhoz:

\begin{Shaded}
\begin{Highlighting}[]
\KeywordTok{SELECT}\NormalTok{ c.asciiname, ic.Code}
\KeywordTok{INTO}\NormalTok{ #euCountriesIsos}
\KeywordTok{FROM}\NormalTok{ Iso2Codes ic}
\KeywordTok{INNER} \KeywordTok{JOIN}\NormalTok{ Countries c }\KeywordTok{on}\NormalTok{ ic.Name }\OperatorTok{=}\NormalTok{ c.country}
\end{Highlighting}
\end{Shaded}

Városok kiválasztása:

\begin{Shaded}
\begin{Highlighting}[]
\KeywordTok{CREATE} \KeywordTok{TABLE}\NormalTok{ #selectedCities (}
\KeywordTok{ID} \DataTypeTok{int} \KeywordTok{NOT} \KeywordTok{NULL}\NormalTok{ IDENTITY(}\DecValTok{1}\NormalTok{,}\DecValTok{1}\NormalTok{) }\KeywordTok{PRIMARY} \KeywordTok{KEY}\NormalTok{,}
\NormalTok{asciiname }\DataTypeTok{VARCHAR}\NormalTok{(}\DecValTok{255}\NormalTok{) }\KeywordTok{NOT} \KeywordTok{NULL} \KeywordTok{UNIQUE}\NormalTok{,}
\NormalTok{Code }\DataTypeTok{CHAR}\NormalTok{(}\DecValTok{2}\NormalTok{) }\KeywordTok{NOT} \KeywordTok{NULL}
\NormalTok{)}

\KeywordTok{INSERT} \KeywordTok{INTO}\NormalTok{ #selectedCities}
\KeywordTok{SELECT}\NormalTok{ tmp.asciiname, tmp.Code}
\KeywordTok{FROM}\NormalTok{ (}
    \KeywordTok{SELECT} \KeywordTok{DISTINCT}\NormalTok{ TOP }\DecValTok{15}\NormalTok{ Code, asciiname }
    \KeywordTok{FROM}\NormalTok{ #euCountriesIsos eci }
    \KeywordTok{WHERE}\NormalTok{ (}\FunctionTok{ABS}\NormalTok{(}\FunctionTok{CAST}\NormalTok{((CHECKSUM(}\OperatorTok{*}\NormalTok{) }\OperatorTok{*}\NormalTok{ RAND()) }\KeywordTok{as} \DataTypeTok{int}\NormalTok{)) % }\DecValTok{100}\NormalTok{) }\OperatorTok{<} \DecValTok{30}\NormalTok{) tmp}
\KeywordTok{UNION}
\NormalTok{    (}\KeywordTok{SELECT} \StringTok{'Luton'}\NormalTok{ asciiname, }\StringTok{'GB'}\NormalTok{ Code)}
\end{Highlighting}
\end{Shaded}

A \emph{TABLESAMPLE} utasítás pontatlansága miatt a \emph{WHERE
(ABS(CAST((CHECKSUM(}) * RAND()) as int)) \% 100) \textless{} 30*
segítségével veszünk véletlenszerű mintát a \emph{\#euCountriesIsos}
táblából.

Ezek után generálunk 2500, hatjegyű \emph{CustomerID}-t:

\begin{Shaded}
\begin{Highlighting}[]
\KeywordTok{WITH}
\NormalTok{  l0(i) }\KeywordTok{AS}\NormalTok{ (}\KeywordTok{SELECT} \DecValTok{0} \KeywordTok{UNION} \KeywordTok{ALL} \KeywordTok{SELECT} \DecValTok{0}\NormalTok{),}
\NormalTok{  l1(i) }\KeywordTok{AS}\NormalTok{ (}\KeywordTok{SELECT} \DecValTok{0} \KeywordTok{FROM}\NormalTok{ l0 a, l0 b),}
\NormalTok{  l2(i) }\KeywordTok{AS}\NormalTok{ (}\KeywordTok{SELECT} \DecValTok{0} \KeywordTok{FROM}\NormalTok{ l1 a, l1 b),}
\NormalTok{  l3(i) }\KeywordTok{AS}\NormalTok{ (}\KeywordTok{SELECT} \DecValTok{0} \KeywordTok{FROM}\NormalTok{ l2 a, l2 b),}
\NormalTok{  l4(i) }\KeywordTok{AS}\NormalTok{ (}\KeywordTok{SELECT} \DecValTok{0} \KeywordTok{FROM}\NormalTok{ l3 a, l3 b),}
\NormalTok{  numbers(i) }\KeywordTok{AS}\NormalTok{ (}\KeywordTok{SELECT} \FunctionTok{ROW_NUMBER}\NormalTok{() }\KeywordTok{OVER}\NormalTok{(}\KeywordTok{ORDER} \KeywordTok{BY}\NormalTok{ (}\KeywordTok{SELECT} \DecValTok{0}\NormalTok{)) }\KeywordTok{FROM}\NormalTok{ l4)}
\KeywordTok{SELECT} \KeywordTok{DISTINCT}\NormalTok{ TOP(}\DecValTok{2500}\NormalTok{) (CEILING(RAND(CHECKSUM(NEWID()))}\OperatorTok{*}\DecValTok{899999}\NormalTok{)}\OperatorTok{+}\DecValTok{100000}\NormalTok{) CustomerID }
\KeywordTok{INTO}\NormalTok{ #cids }
\KeywordTok{FROM}\NormalTok{ numbers }
\KeywordTok{ORDER} \KeywordTok{BY}\NormalTok{ CustomerID}
\end{Highlighting}
\end{Shaded}

\emph{CustomerID}-khoz \emph{CCountry} azonosítókat rendelünk:

\begin{Shaded}
\begin{Highlighting}[]
\KeywordTok{SELECT}\NormalTok{ cp.CustomerID CustomerID, sc.Code CCountry}
\KeywordTok{INTO}\NormalTok{ #cidPairs}
\KeywordTok{FROM}\NormalTok{ (}\KeywordTok{SELECT}\NormalTok{ cids.CustomerID, (CEILING(RAND(CHECKSUM(NEWID()))}\OperatorTok{*}\NormalTok{(}\KeywordTok{SELECT} \FunctionTok{COUNT}\NormalTok{(}\OperatorTok{*}\NormalTok{) }\KeywordTok{FROM}\NormalTok{ #selectedCities))) CountryID }\KeywordTok{FROM}\NormalTok{ #cids cids) cp}
\KeywordTok{INNER} \KeywordTok{JOIN}\NormalTok{ #selectedCities sc }\KeywordTok{on}\NormalTok{ cp.CountryID }\OperatorTok{=}\NormalTok{ sc.}\KeywordTok{ID}
\KeywordTok{ORDER} \KeywordTok{BY}\NormalTok{ CustomerID}
\end{Highlighting}
\end{Shaded}

\hypertarget{bookings-feltuxf6ltuxe9se}{%
\subsection{Bookings feltöltése}\label{bookings-feltuxf6ltuxe9se}}

\emph{Bookings} tábla létrehozása:

\begin{Shaded}
\begin{Highlighting}[]
\KeywordTok{CREATE} \KeywordTok{TABLE}\NormalTok{ Bookings(}
\NormalTok{[BookingID] }\DataTypeTok{int} \KeywordTok{NOT} \KeywordTok{NULL}\NormalTok{ IDENTITY(}\DecValTok{1}\NormalTok{,}\DecValTok{1}\NormalTok{) }\KeywordTok{PRIMARY} \KeywordTok{KEY}\NormalTok{,}
\NormalTok{[CustomerID] }\DataTypeTok{int} \KeywordTok{NOT} \KeywordTok{NULL}\NormalTok{,}
\NormalTok{[CCountry] }\DataTypeTok{varchar}\NormalTok{(}\DecValTok{2}\NormalTok{) }\KeywordTok{NOT} \KeywordTok{NULL}\NormalTok{,}
\NormalTok{[DepartureStation] }\DataTypeTok{varchar}\NormalTok{(}\DecValTok{30}\NormalTok{) }\KeywordTok{NOT} \KeywordTok{NULL}\NormalTok{,}
\NormalTok{[}\DataTypeTok{Date}\NormalTok{] datetime }\KeywordTok{NOT} \KeywordTok{NULL} \KeywordTok{DEFAULT}\NormalTok{ (GETDATE()),}
\NormalTok{[Price] money }\KeywordTok{NOT} \KeywordTok{NULL}\NormalTok{,}
\NormalTok{[Seats] }\DataTypeTok{int} \KeywordTok{NOT} \KeywordTok{NULL}
\NormalTok{)}
\end{Highlighting}
\end{Shaded}

Egy kezdeti adathalmaz létrehozása:

\begin{Shaded}
\begin{Highlighting}[]
\KeywordTok{WITH}
\NormalTok{  l0(i) }\KeywordTok{AS}\NormalTok{ (}\KeywordTok{SELECT} \DecValTok{0} \KeywordTok{UNION} \KeywordTok{ALL} \KeywordTok{SELECT} \DecValTok{0}\NormalTok{),}
\NormalTok{  l1(i) }\KeywordTok{AS}\NormalTok{ (}\KeywordTok{SELECT} \DecValTok{0} \KeywordTok{FROM}\NormalTok{ l0 a, l0 b),}
\NormalTok{  l2(i) }\KeywordTok{AS}\NormalTok{ (}\KeywordTok{SELECT} \DecValTok{0} \KeywordTok{FROM}\NormalTok{ l1 a, l1 b),}
\NormalTok{  l3(i) }\KeywordTok{AS}\NormalTok{ (}\KeywordTok{SELECT} \DecValTok{0} \KeywordTok{FROM}\NormalTok{ l2 a, l2 b),}
\NormalTok{  numbers(i) }\KeywordTok{AS}\NormalTok{ (}\KeywordTok{SELECT} \FunctionTok{ROW_NUMBER}\NormalTok{() }\KeywordTok{OVER}\NormalTok{(}\KeywordTok{ORDER} \KeywordTok{BY}\NormalTok{ (}\KeywordTok{SELECT} \DecValTok{0}\NormalTok{)) }\KeywordTok{FROM}\NormalTok{ l3)}
\KeywordTok{SELECT}\NormalTok{ TOP(}\DecValTok{1000000}\NormalTok{)}
\NormalTok{    cp.CustomerID,}
\NormalTok{    cp.CCountry,}
\NormalTok{    sc.asciiname DepartureStation,}
\NormalTok{    (DATEADD(ms,RAND(CHECKSUM(NEWID())) }\OperatorTok{*} \DecValTok{86400000}\NormalTok{,DATEADD(d,RAND(CHECKSUM(NEWID()))}\OperatorTok{*}\DecValTok{365}\NormalTok{,}\StringTok{'2019-01-01'}\NormalTok{))) [}\DataTypeTok{Date}\NormalTok{],}
\NormalTok{    (}\FunctionTok{ROUND}\NormalTok{(RAND(CHECKSUM(NEWID()))}\OperatorTok{*}\DecValTok{200}\OperatorTok{+}\DecValTok{10}\NormalTok{,}\DecValTok{2}\NormalTok{)) Price,}
\NormalTok{    (CEILING(RAND(CHECKSUM(NEWID()))}\OperatorTok{*}\DecValTok{50}\NormalTok{)) Seats}
\KeywordTok{INTO}\NormalTok{ #bookingsAll}
\KeywordTok{from}\NormalTok{ numbers}
\KeywordTok{CROSS}\NormalTok{ APPLY (}
   \KeywordTok{SELECT} 
\NormalTok{    cp.CustomerID, }
\NormalTok{    cp.CCountry,}
\NormalTok{    (}
        \KeywordTok{SELECT}\NormalTok{ CEILING(RAND(CHECKSUM(NEWID()))}\OperatorTok{*}\NormalTok{(}\KeywordTok{SELECT} \FunctionTok{COUNT}\NormalTok{(}\OperatorTok{*}\NormalTok{) }
        \KeywordTok{FROM}\NormalTok{ #selectedCities))}
\NormalTok{    ) DepartureStationID}
    \KeywordTok{FROM}\NormalTok{ #cidPairs }\KeywordTok{AS}\NormalTok{ cp}
\NormalTok{) }\KeywordTok{AS}\NormalTok{ cp}
\KeywordTok{INNER} \KeywordTok{JOIN}\NormalTok{ #selectedCities sc }
    \KeywordTok{ON}\NormalTok{ sc.}\KeywordTok{ID} \OperatorTok{=}\NormalTok{ cp.DepartureStationID}
\end{Highlighting}
\end{Shaded}

Az adathalmazból véletlenszerűen kiválasztásra kerül tízezer bejegyzés a
végleges \emph{Bookings} táblába:

\begin{Shaded}
\begin{Highlighting}[]

\KeywordTok{INSERT} \KeywordTok{INTO}\NormalTok{ Bookings}
\KeywordTok{SELECT}\NormalTok{ TOP(}\DecValTok{10000}\NormalTok{) }\OperatorTok{*}
\KeywordTok{FROM}\NormalTok{ #bookingsAll}
\KeywordTok{WHERE}\NormalTok{ (}\FunctionTok{ABS}\NormalTok{(}\FunctionTok{CAST}\NormalTok{((CHECKSUM(}\OperatorTok{*}\NormalTok{) }\OperatorTok{*}\NormalTok{ RAND()) }\KeywordTok{as} \DataTypeTok{int}\NormalTok{)) % }\DecValTok{100}\NormalTok{) }\OperatorTok{<} \DecValTok{30}
\end{Highlighting}
\end{Shaded}

\hypertarget{indexek-luxe9trehozuxe1sa}{%
\subsection{Indexek létrehozása}\label{indexek-luxe9trehozuxe1sa}}

\emph{BookingID} mezőn alapértelmezetten létrejön egy clustered index a
\emph{PRIMARY KEY} feltétel miatt

Non Clustered Index a CCountry és CustomerID párokon:

\begin{Shaded}
\begin{Highlighting}[]
\KeywordTok{CREATE}\NormalTok{ NONCLUSTERED }\KeywordTok{INDEX}\NormalTok{ NC_Bookings_countryCid }\KeywordTok{ON}\NormalTok{ [dbo].[Bookings]}
\NormalTok{(}
\NormalTok{    [CCountry] }\KeywordTok{ASC}\NormalTok{,}
\NormalTok{    [CustomerID] }\KeywordTok{ASC}
\NormalTok{)}
\end{Highlighting}
\end{Shaded}

Ezen kívűl a feladat lekérdezéseihez a Database Engine Tuning Advisor
által ajánlott optimális non clustered indexek:

\begin{Shaded}
\begin{Highlighting}[]
\KeywordTok{CREATE}\NormalTok{ NONCLUSTERED }\KeywordTok{INDEX}\NormalTok{ NC_Bookings_countryCidDate }\KeywordTok{ON}\NormalTok{ [dbo].[Bookings]}
\NormalTok{(}
\NormalTok{    [}\DataTypeTok{Date}\NormalTok{] }\KeywordTok{ASC}\NormalTok{,}
\NormalTok{    [CCountry] }\KeywordTok{ASC}\NormalTok{,}
\NormalTok{    [CustomerID] }\KeywordTok{ASC}
\NormalTok{)}
\NormalTok{INCLUDE([Price])}

\KeywordTok{CREATE}\NormalTok{ NONCLUSTERED }\KeywordTok{INDEX}\NormalTok{ NC_Bookings_countryDateStation }\KeywordTok{ON}\NormalTok{ [dbo].[Bookings]}
\NormalTok{(}
\NormalTok{    [DepartureStation] }\KeywordTok{ASC}\NormalTok{,}
\NormalTok{    [}\DataTypeTok{Date}\NormalTok{] }\KeywordTok{ASC}\NormalTok{,}
\NormalTok{    [CCountry] }\KeywordTok{ASC}
\NormalTok{)}
\NormalTok{INCLUDE([Price])}
\end{Highlighting}
\end{Shaded}

\hypertarget{lekuxe9rdezuxe9sek}{%
\section{Lekérdezések}\label{lekuxe9rdezuxe9sek}}

\hypertarget{feladat---2019-muxe1jusi-megrendeluxe9sek-csuxf6kkenux151-sorrendben}{%
\paragraph{2. feladat - 2019 májusi megrendelések csökkenő
sorrendben}\label{feladat---2019-muxe1jusi-megrendeluxe9sek-csuxf6kkenux151-sorrendben}}

\begin{Shaded}
\begin{Highlighting}[]
\KeywordTok{SELECT}\NormalTok{ b.CustomerID, b.CCountry , }\FunctionTok{SUM}\NormalTok{(b.Price) }\StringTok{'Total Amount'}
\KeywordTok{FROM}\NormalTok{ Bookings b}
\KeywordTok{WHERE}\NormalTok{ b.[}\DataTypeTok{Date}\NormalTok{] }\OperatorTok{>=} \StringTok{'2019-05-01'} \KeywordTok{AND}\NormalTok{ b.[}\DataTypeTok{Date}\NormalTok{] }\OperatorTok{<} \StringTok{'2019-06-01'}
\KeywordTok{GROUP} \KeywordTok{BY}\NormalTok{ b.CustomerID,b.CCountry}
\KeywordTok{ORDER} \KeywordTok{BY} \StringTok{'Total Amount'} \KeywordTok{DESC}
\end{Highlighting}
\end{Shaded}

\hypertarget{feladat---muxe1sodik-negyeduxe9v-luton-buxf3l-induluxf3-megrendeluxe9seinek-uxf6sszegei}{%
\paragraph{3. feladat - Második negyedév Luton-ból induló
megrendeléseinek
összegei}\label{feladat---muxe1sodik-negyeduxe9v-luton-buxf3l-induluxf3-megrendeluxe9seinek-uxf6sszegei}}

\begin{Shaded}
\begin{Highlighting}[]
\KeywordTok{SELECT}\NormalTok{ b.CCountry Country,}
\NormalTok{    ISNULL(}\FunctionTok{SUM}\NormalTok{(}\ControlFlowTok{case} \ControlFlowTok{when}\NormalTok{ DATEPART(m,b.[}\DataTypeTok{Date}\NormalTok{]) }\OperatorTok{=} \DecValTok{4} \ControlFlowTok{THEN}\NormalTok{ b.Price }\ControlFlowTok{END}\NormalTok{),}\DecValTok{0}\NormalTok{) }\StringTok{'2019-04'}\NormalTok{,}
\NormalTok{    ISNULL(}\FunctionTok{SUM}\NormalTok{(}\ControlFlowTok{case} \ControlFlowTok{when}\NormalTok{ DATEPART(m,b.[}\DataTypeTok{Date}\NormalTok{]) }\OperatorTok{=} \DecValTok{5} \ControlFlowTok{THEN}\NormalTok{ b.Price }\ControlFlowTok{END}\NormalTok{),}\DecValTok{0}\NormalTok{) }\StringTok{'2019-05'}\NormalTok{,}
\NormalTok{    ISNULL(}\FunctionTok{SUM}\NormalTok{(}\ControlFlowTok{case} \ControlFlowTok{when}\NormalTok{ DATEPART(m,b.[}\DataTypeTok{Date}\NormalTok{]) }\OperatorTok{=} \DecValTok{6} \ControlFlowTok{THEN}\NormalTok{ b.Price }\ControlFlowTok{END}\NormalTok{),}\DecValTok{0}\NormalTok{) }\StringTok{'2019-06'}
\KeywordTok{FROM}\NormalTok{ Bookings b}
\KeywordTok{WHERE}\NormalTok{ b.DepartureStation }\OperatorTok{=} \StringTok{'Luton'} \KeywordTok{AND}\NormalTok{ b.[}\DataTypeTok{Date}\NormalTok{] }\OperatorTok{>=} \StringTok{'2019-04-01'} \KeywordTok{AND}\NormalTok{ b.[}\DataTypeTok{Date}\NormalTok{] }\OperatorTok{<} \StringTok{'2019-07-01'}
\KeywordTok{GROUP} \KeywordTok{BY}\NormalTok{ b.CCountry}
\KeywordTok{ORDER} \KeywordTok{BY} \FunctionTok{SUM}\NormalTok{(b.Price) }\KeywordTok{DESC}
\end{Highlighting}
\end{Shaded}

Megvalósítható \emph{PIVOT} használatával is, ez azonban csak kevésbé
rugalmas megjelenítést tesz lehetővé

\begin{Shaded}
\begin{Highlighting}[]
\KeywordTok{SELECT} \OperatorTok{*} \KeywordTok{FROM} 
\NormalTok{(}\KeywordTok{SELECT}\NormalTok{ b.CCountry Country, FORMAT(b.[}\DataTypeTok{Date}\NormalTok{],}\StringTok{'yyyy-MM'}\NormalTok{) dpart, b.Price}
    \KeywordTok{FROM}\NormalTok{ Bookings b}
    \KeywordTok{WHERE}\NormalTok{ b.DepartureStation }\OperatorTok{=} \StringTok{'Luton'} \KeywordTok{AND}\NormalTok{ b.[}\DataTypeTok{Date}\NormalTok{] }\OperatorTok{>=} \StringTok{'2019-04-01'} \KeywordTok{AND}\NormalTok{ b.[}\DataTypeTok{Date}\NormalTok{] }\OperatorTok{<} \StringTok{'2019-07-01'}\NormalTok{) x}
\NormalTok{PIVOT (}
\FunctionTok{SUM}\NormalTok{(x.Price)}
\ControlFlowTok{FOR}\NormalTok{ dpart }\KeywordTok{IN}\NormalTok{ ([}\DecValTok{2019}\OperatorTok{-}\DecValTok{04}\NormalTok{],[}\DecValTok{2019}\OperatorTok{-}\DecValTok{05}\NormalTok{],[}\DecValTok{2019}\OperatorTok{-}\DecValTok{06}\NormalTok{])}
\NormalTok{) PivotTable;}
\end{Highlighting}
\end{Shaded}

\hypertarget{feladat---ruxe9giuxf3nkuxe9nti-uxfcgyfuxe9l-vuxe1suxe1rluxe1si-szuxe1mok-kimutatuxe1sa}{%
\paragraph{4. feladat - Régiónkénti ügyfél vásárlási számok
kimutatása}\label{feladat---ruxe9giuxf3nkuxe9nti-uxfcgyfuxe9l-vuxe1suxe1rluxe1si-szuxe1mok-kimutatuxe1sa}}

\begin{Shaded}
\begin{Highlighting}[]
\KeywordTok{SELECT}\NormalTok{ cntQry.CCountry Country, }
    \FunctionTok{COUNT}\NormalTok{(}\ControlFlowTok{case} \ControlFlowTok{when}\NormalTok{ cntQry.buyCnt }\OperatorTok{=} \DecValTok{1} \ControlFlowTok{THEN} \DecValTok{1} \ControlFlowTok{ELSE} \KeywordTok{null} \ControlFlowTok{END}\NormalTok{) [}\DecValTok{1}\NormalTok{], }
    \FunctionTok{COUNT}\NormalTok{(}\ControlFlowTok{case} \ControlFlowTok{when}\NormalTok{ cntQry.buyCnt }\OperatorTok{=} \DecValTok{2} \ControlFlowTok{THEN} \DecValTok{1} \ControlFlowTok{ELSE} \KeywordTok{null} \ControlFlowTok{END}\NormalTok{) [}\DecValTok{2}\NormalTok{], }
    \FunctionTok{COUNT}\NormalTok{(}\ControlFlowTok{case} \ControlFlowTok{when}\NormalTok{ cntQry.buyCnt }\OperatorTok{=} \DecValTok{3} \ControlFlowTok{THEN} \DecValTok{1} \ControlFlowTok{ELSE} \KeywordTok{null} \ControlFlowTok{END}\NormalTok{) [}\DecValTok{3}\NormalTok{], }
    \FunctionTok{COUNT}\NormalTok{(}\ControlFlowTok{case} \ControlFlowTok{when}\NormalTok{ cntQry.buyCnt }\KeywordTok{BETWEEN} \DecValTok{4} \KeywordTok{AND} \DecValTok{9} \ControlFlowTok{THEN} \DecValTok{1} \ControlFlowTok{ELSE} \KeywordTok{null} \ControlFlowTok{END}\NormalTok{) [}\DecValTok{4}\OperatorTok{-}\DecValTok{10}\NormalTok{], }
    \FunctionTok{COUNT}\NormalTok{(}\ControlFlowTok{case} \ControlFlowTok{when}\NormalTok{ cntQry.buyCnt }\KeywordTok{BETWEEN} \DecValTok{10} \KeywordTok{AND} \DecValTok{100} \ControlFlowTok{THEN} \DecValTok{1} \ControlFlowTok{ELSE} \KeywordTok{null} \ControlFlowTok{END}\NormalTok{) [}\DecValTok{10}\OperatorTok{-}\DecValTok{100}\NormalTok{]}
\KeywordTok{FROM}\NormalTok{ (}
    \KeywordTok{SELECT} \FunctionTok{COUNT}\NormalTok{(b.BookingID) buyCnt, b.CustomerID, b.CCountry }
    \KeywordTok{FROM}\NormalTok{ Bookings b }
    \KeywordTok{GROUP} \KeywordTok{BY}\NormalTok{ b.CustomerID,b.CCountry}
\NormalTok{    ) cntQry}
\KeywordTok{GROUP} \KeywordTok{BY}\NormalTok{ cntQry.CCountry}
\KeywordTok{ORDER} \KeywordTok{BY} \FunctionTok{SUM}\NormalTok{(cntQry.buyCnt)}
\end{Highlighting}
\end{Shaded}

\hypertarget{feladat---helymegtakaruxedtuxe1s-az-indexek-eldobuxe1suxe1val}{%
\section{6. feladat - Helymegtakarítás az indexek
eldobásával}\label{feladat---helymegtakaruxedtuxe1s-az-indexek-eldobuxe1suxe1val}}

Az indexek méretét több dolog is befolyásolja, így a konkrét tábla és
indexek ismerete nélkül nem lehet meghatározni a helymegtakarítást.
Befolyásoló szempontok lehetnek például a tábla mérete, a pageek
fill-factora vagy indexben használt oszlopok típusai és száma. További
nehezen meghatározható tényező a tömörített indexek és LOB-ok tárhely
igénye

Az indexek által foglalt hely lekérdezhető az alábbi
queryvel{[}\^{}3{]}: {[}\^{}3{]}: \emph{} helyettesítendő a keresett
tábla nevével

\begin{Shaded}
\begin{Highlighting}[]
\KeywordTok{SELECT}
\NormalTok{OBJECT_NAME(i.OBJECT_ID) }\KeywordTok{AS}\NormalTok{ TableName,}
\NormalTok{i.name }\KeywordTok{AS}\NormalTok{ IndexName,}
\NormalTok{i.index_id }\KeywordTok{AS}\NormalTok{ IndexID,}
\DecValTok{8} \OperatorTok{*} \FunctionTok{SUM}\NormalTok{(a.used_pages) }\KeywordTok{AS} \StringTok{'Indexsize(KB)'}
\KeywordTok{FROM}\NormalTok{ sys.}\KeywordTok{indexes} \KeywordTok{AS}\NormalTok{ i}
\KeywordTok{JOIN}\NormalTok{ sys.}\KeywordTok{partitions} \KeywordTok{AS}\NormalTok{ p }\KeywordTok{ON}\NormalTok{ p.OBJECT_ID }\OperatorTok{=}\NormalTok{ i.OBJECT_ID }\KeywordTok{AND}\NormalTok{ p.index_id }\OperatorTok{=}\NormalTok{ i.index_id}
\KeywordTok{JOIN}\NormalTok{ sys.allocation_units }\KeywordTok{AS}\NormalTok{ a }\KeywordTok{ON}\NormalTok{ a.container_id }\OperatorTok{=}\NormalTok{ p.partition_id}
\KeywordTok{WHERE}\NormalTok{ OBJECT_NAME(i.OBJECT_ID) }\OperatorTok{=} \OperatorTok{<}\NormalTok{TÁBLA}\OperatorTok{>}
\KeywordTok{GROUP} \KeywordTok{BY}\NormalTok{ i.OBJECT_ID,i.index_id,i.name}
\KeywordTok{ORDER} \KeywordTok{BY}\NormalTok{ OBJECT_NAME(i.OBJECT_ID),i.index_id}
\end{Highlighting}
\end{Shaded}

\hypertarget{section}{%
\section{}\label{section}}

\hypertarget{random-jegyzetek}{%
\section{random jegyzetek}\label{random-jegyzetek}}

\begin{quote}
SNAPShOT ISOLATION ON tempdb-be az update előtti adatok bekerülnek egy
row version számmal p: nincs lock
\end{quote}

\begin{verbatim}
c: tempdb igénybevétel nő
\end{verbatim}

\begin{quote}
Read committed Snapshot On olvasni cak commitelt adatokból
\end{quote}

\begin{quote}
Mire érdemes figyelni: több update-nél UPDLock hinttel zárható az
updatelt sor kiadott update-eknél van e select updlockkal, vagy
hibakezelés (begin try/end try)
\end{quote}

\begin{quote}
(INDEX) 300MB tábla, 2 nonCL, 1 CL index eldobás hatása a méretre
(\%ban)
\end{quote}

\begin{verbatim}
[BookingID] int 4b
[CustomerID] int 4b
[CCountry] varchar(2) 4b (2+2)
[DepartureStation] varchar(30) 32b (30+2)
[Date] datetime 8b
[Price] money 8b
[Seats] int 4b 
Row req: 70b (28 + [2+2*2+32] + 4 rowHeader)

Clustered a BookingID-n automatikusan
Row req 11b (4+1 rowHeader+6 childID)

NC 2 szűk DepState & CID
Row reqs 43b  <- 11b (4+1+6) & 32b (2+30)

NEM számol vele: page veszteség, non-leaf page méret, tömörítés

arányok: 70:54 -> 300M/124*70 -> 164M adat -> 45% csökkenés
\end{verbatim}

\renewcommand*\contentsname{Tartalomjegyzék}
{
\hypersetup{linkcolor=}
\setcounter{tocdepth}{3}
\tableofcontents
}

\end{document}
